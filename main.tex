%!TEX TS-program = xelatex

%% If you need to pass whatever options to xcolor
\PassOptionsToPackage{dvipsnames}{xcolor}


\documentclass[10pt,a4paper,ragged2e,withhyper]{altacv}
%% You MUST compile with XeLaTeX or LuaLaTeX if you want to use academicons.

\newenvironment{sloppypar*}{\sloppy\ignorespaces}{\par}

% Change the page layout if you need to
\geometry{left=1.2cm,right=1.2cm,top=1cm,bottom=1cm,columnsep=0.75cm}

% The paracol package lets you typeset columns of text in parallel
\usepackage{paracol}
\usepackage{xeCJK}
\usepackage{fontspec}


\ifxetexorluatex
  % If using xelatex or lualatex:
  \setmainfont{Roboto Slab}
  \setsansfont{Lato}
  \renewcommand{\familydefault}{\sfdefault}

\fi

% Fork (before v1.6.5a): Change the color codes to test your personal variant on any mode
\ifdarkmode%
  \definecolor{PrimaryColor}{HTML}{C69749}
  \definecolor{SecondaryColor}{HTML}{D49B54}
  \definecolor{ThirdColor}{HTML}{1877E8}
  \definecolor{BodyColor}{HTML}{ABABAB}
  \definecolor{EmphasisColor}{HTML}{ABABAB}
  \definecolor{BackgroundColor}{HTML}{191919}
\else%
  \definecolor{PrimaryColor}{HTML}{001F5A}
  \definecolor{SecondaryColor}{HTML}{0039AC}
  \definecolor{ThirdColor}{HTML}{F3890B}
  \definecolor{BodyColor}{HTML}{666666}
  \definecolor{EmphasisColor}{HTML}{2E2E2E}
  \definecolor{BackgroundColor}{HTML}{E2E2E2}
\fi%

\colorlet{name}{PrimaryColor}
\colorlet{tagline}{SecondaryColor}
\colorlet{heading}{PrimaryColor}
\colorlet{headingrule}{ThirdColor}
\colorlet{subheading}{SecondaryColor}
\colorlet{accent}{SecondaryColor}
\colorlet{emphasis}{EmphasisColor}
\colorlet{body}{BodyColor}
\pagecolor{BackgroundColor}


\renewcommand{\namefont}{\Huge\rmfamily\bfseries}
\renewcommand{\personalinfofont}{\small\bfseries}
\renewcommand{\cvsectionfont}{\LARGE\rmfamily\bfseries}
\renewcommand{\cvsubsectionfont}{\large\bfseries}
\renewcommand{\itemmarker}{{\small\textbullet}}
\renewcommand{\ratingmarker}{\faCircle}
\begin{document}
    \name{Steve Zeng}
    \tagline{}

    \personalinfo{
        \email{szeng2525@gmail.com}\smallskip
        \email{szeng5@uiowa.edu}\smallskip
        \phone{+1-707-774-9127}
        \location{Iowa, United States}\smallskip
        \github{mathemate}
        \homepage{szeng.pages.dev}
    }
    
    \makecvheader
    \columnratio{0.4}

    \begin{paracol}{2}
        \cvsection{About}
        I am an aspiring student who enjoys programming, mathematics, and chemistry who hopes
        to one day revolutionize the world with new innovations by 2050.
        \bigskip
        
        \cvsection{Languages}
        \cvlang{English}{Native}
        \bigskip
   
        \cvlang{日本}{N3}
        \bigskip
        
        \cvlang{中文}{HSK4}

        \cvsection{Labels}
            \begin{sloppypar*}
            	\cvtags{  Julia, R, Matlab, Python, Javascript, C, C++}
                
                \cvtags{GCP, AWS, Linux}
                
				\cvtags{Github}

				\cvtags {Tensorflow, Pytorch, LaTeX}

            \end{sloppypar*}
       
       
        \switchcolumn

        \cvsection{Projects}
        \cvevent{Light The World(LTW)} {}{01 2023 -- 08 2023}{}
        \begin{itemize}
            \item Used Tensorflow to design different models for different tasks. 
            \item Used a sparesley entropy loss and LSTM for an emotion detector, which detects the emotion based on a string of text,
            \item Fine-tuned using Huggingfaces Pipelines 13B parameter model to be specific characters(Character AI Engine)
            \item LTW Character DEMO: A ReactJS website which uses DJango backend. 
            \item API Server runs a fine-tuned AI models to act as a character and respond to the user. 
        \end{itemize}
        \divider
        
        \cvevent{MATHEMATE(Self-Projects)}{\cvreference{\faGithub}{https://github.com/mathemate}\cvreference{| \faGlobe}{https://szeng.pages.dev}}{06 2023 -- 12 2023}{}
        \begin{itemize}
            \item Main Website: Used front-end web technologies like React to design my personal website, which recieves 1K+ views and 100+ unique visitors.
            \item Differential Equation Solver: Used Julia to design a differential equation solver, which can solve many various types of differential equations,
                  and solve specific types of PDEs like heat equation and the wave equation.
            


        \end{itemize}
        \divider


        \cvsection{Education}
            \cvevent{University of Iowa}{Student}{08 2022 -- 05 2025}{Iowa City, United States}
            \begin{itemize}
                \item GPA: 3.85
                \item Courses: Introduction to Computer Engineering, 
                               Computer Science I, 
                               Discrete Structures, 
                               Data Structures,
                               Calculus I and II, 
                               Linear Algebra, 
                               Differential Equations, 
                               Principles of Chemistry, 
                               Organic Chemistry, 
                               Physical Chemistry 
            \end{itemize}

            \cvsection{Awards}
            \cvevent{United States of America Chemistry Olympiad National Qualifier 2023}{}{}{Iowa , United States}
            \begin{itemize}
                \item Qualfied for the National Exam, which is the around 1,000 students out of 16,000 students.
                \item 99th percentile in the state of Iowa
                \item Partook in a 4.5 hour multi-part exam, which tested labotoratory skills, and knowledge of 
                   organic chemistry and general chemistry.
            \end{itemize}
            \cvevent{TSA Coding Competition 2022}{}{}{Iowa, United States}
            \begin{itemize}
                \item 2nd place team in the state of Iowa.
                \item Wrote algorithms in Java to solve challenging math problems.
            \end{itemize}

    
    \end{paracol}
 

\end{document}